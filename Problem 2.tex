\documentclass[11pt]{amsart}


%The package below let's us add todos to the document in a way that will stand out at a glance.  It will also list all of the current things to do, with their respective page numbers on the first page of the document.  You can add a todo like this:
%
%		Text text text \todo{change this text to something useful!}
%
%Among other things, you can edit the color of a todo:
%
%		\todo[color=\ltblue]{this todo will have a light blue background!}
%
%And you can include the todo as a line in the text, rather than in the margins:
%
%		\todo[inline]{this will appear in the middle of the page}

\usepackage[colorinlistoftodos, textsize=tiny]{todonotes}
\def\ltblue{blue!20!white}
\def\ltgreen{green!20!white}

\usepackage{amsmath,amssymb,amsthm, MnSymbol}


\newtheorem{thm}{Theorem}[section]
\newtheorem{lem}[thm]{Lemma}
\newtheorem{prop}[thm]{Proposition}
\newtheorem{cor}[thm]{Corollary}
\newtheorem{conj}[thm]{Conjecture}

\theoremstyle{definition}
\newtheorem{defn}[thm]{Definition}
\newtheorem{rem}[thm]{Remark}


\makeatletter
\providecommand\@dotsep{5}
\def\listtodoname{List of Todos}
\def\listoftodos{\@starttoc{tdo}\listtodoname}
\makeatother


\begin{document}

%This adds the list of todos to the first page.
\listoftodos
\newpage


\section{Notation and Definitions}

Let $G\subset \mathfrak{S}_n$, let $x,y\in 2^{[n]}$, and let $1\le r\le n$.  For $r\le i\le n$, let $V_i^{(r)}$ be the $\mathbb{R}$-vector space generated by the basis

$$\{e_{(x,y)}\}_{x\subset y, |y| = i, |x| = i-r}$$

Note that $G$ acts on this space with the action $\sigma e_{(x,y)} = e_{(\sigma x,\sigma y)}$.  Let $\left(V_i^{(r)}\right)^G$ be the subspace of $V_i^{(r)}$ that is invariant under this action.  Write $q_i^r = \dim\left(\left(V_i^{(r)}\right)^G\right)$.  When $r$ is understood to be 1, we will often simply write $q_i$.

Let $p_i$ be defined as by Pak and Panova, that is

$$p_i = \sum_{Gy, |y| = i} \nu(Gy)$$

where $\nu(Gy)$ is the number of covering relations $Gy \cdot> Gx$.\todo{what's the latex command for the dot?}




\section{Current Goals}
Our main goal is describe the groups $G\subset \mathfrak{S}_n$ such that for $2^{[n]}/G$ we have $q_i = p_i$.  So far we know that this is true for $G = \{e\}$, $G = \mathfrak{S}_n$, and $G = \mathfrak{S}_k \wreath \mathfrak{S}_l$ (this is equivalent to Proposition \ref{prop:pak_panova_comparison}).  Any other groups for which it holds would be good to know about.  In fact, if someone wanted to write a program to compute the sequence $q_i$ for all subgroups of $\mathfrak{S}_4$, $\mathfrak{S}_5$, $\mathfrak{S}_6$, $\mathfrak{S}_7$ and such, that would be awesome!




\section{Conjectures}
Put any conjectures, no matter how wild, here.





\section{Results}



\begin{prop}\label{prop:q_injective_unimodal}
For all $G\subset\mathfrak{S}_n$, $1\le r\le n$, the sequence $q_r^r, q_{r+1}^r,\ldots, q_n^r$ is unimodal and symmetric about $\frac{n+r}{2}$.
\end{prop}

\begin{prop}\label{prop:pak_panova_comparison}
For $G = \mathfrak{S}_k \wreath \mathfrak{S}_l$ and $r = 1$, $q_i = p_i$ for all $1\le i\le n$.
\end{prop}


\begin{lem}\label{lem:U_injective}
$U_i^{(r)}$ is injective for all $i < \frac{n + r}{2}$.
\end{lem}



\begin{lem}\label{lem:U_commutes_with_action}
For all $\sigma\in G, e_{(x,y)}\in V_i^{(r)}$, we have

$$U_i^{(r)}(\sigma(e_{(x,y)})) = \sigma(U_i^{(r)}(e_{(x,y)}))$$

\end{lem}


\begin{lem}\label{lem:G_invariant_basis}
For any group $G\subset \mathfrak{S}_n$ with an action on an $\mathbb{R}$-vector space $V$ with basis $\{v_i\}_{1\le i\le k}$, the $G$-invariant subspace $V^G$ of $V$ has basis

$$\sum_{v_i\in Gv} v_i $$

\noindent where the sum is taken over the orbits $Gv$ of the group action.
\end{lem}



\section{Failed Attempts}
Have an idea that failed?  Show it off here!





\end{document}