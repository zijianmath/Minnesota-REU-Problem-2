%%%%%%%%%%%%%%%%%%%%%
%   AMS packages    %
%%%%%%%%%%%%%%%%%%%%%
\documentclass{amsart}
\usepackage{amsmath}
\usepackage{amsxtra}
\usepackage{amscd}
\usepackage{amsthm}
\usepackage{amsfonts}
\usepackage{amssymb}
\usepackage{eucal}
\usepackage[all]{xy}
\usepackage{graphicx}
\usepackage{comment}
\usepackage{amssymb}

\newtheorem{cor}[subsubsection]{Corollary}
\newtheorem{lem}[subsubsection]{Lemma}
\newtheorem{prop}[subsubsection]{Proposition}
\newtheorem{propconstr}{Proposition-Construction}
\newtheorem{ax}{Axiom}
\newtheorem{conj}{Conjecture}
\newtheorem{thm}[subsubsection]{Theorem}
\newtheorem{defn}[subsubsection]{Definition}
\newtheorem{rem}[subsubsection]{Remark}
\newtheorem{eg}[subsubsection]{Example}
\newtheorem{ex}[subsubsection]{Exercise}
\newtheorem{note}[subsubsection]{Notation}
\newtheorem{alg}[subsubsection]{Algorithm}
\newtheorem{fact}[subsubsection]{Fact}

\newcommand\nc{\newcommand}
\nc\on{\operatorname}
\nc\renc{\renewcommand}
\newcommand\ssec{\subsection}
\newcommand\sssec{\subsubsection}
\newcommand\bO{{\mathbf O}}
\newcommand\CC{{\mathcal C}}
\newcommand\BN{{\mathbb N}}
\newcommand\BC{{\mathbb C}}
\newcommand\BF{{\mathbb F}}
\newcommand\BR{{\mathbb R}}
\newcommand\BQ{{\mathbb Q}}
\newcommand\BBZ{{\mathbb Z}}
\newcommand\uR{\underline{R}}
\newcommand\uZ{\underline{\BBZ}}
\newcommand\CF{{\mathcal F}}
\newcommand\uCF{\underline{{\mathcal F}}}
\newcommand\BZ{{\mathbb Z}}
\newcommand\BA{{\mathbb A}}
\newcommand\BP{{\mathbb P}}
\newcommand\fa{{\mathfrak a}}
\newcommand\fp{{\mathfrak p}}
\newcommand\fq{{\mathfrak q}}
\newcommand\fm{{\mathfrak m}}
\newcommand\pt{\mathrm{pt}}
\nc{\bd}{\mathbf{d}}
\nc{\Hom}{\on{Hom}}
\nc{\End}{\on{End}}
\nc{\Spec}{\on{Spec}}
\nc{\Reg}{\on{Reg}}
\nc{\Specm}{\on{Specm}}
\nc\ol{\overline}
\nc\wt{\widetilde}
\nc{\one}{{\mathbf{1}}}
\renc{\mod}{\on{-mod}}
\newcommand{\id}{\mathrm{id}}
\nc{\ul}{\underline}
\nc{\uHom}{\ul\Hom}
\nc{\tHom}{\ul\uHom}
\nc{\wh}{\widehat}
\nc{\Vect}{\on{Vect}}
\nc{\Res}{\on{Res}}
\nc{\Ind}{\on{Ind}}

\title{Unimodality Ideas}
\author{Aaron Landesman}
\usepackage{amsmath}
\begin{document}

\maketitle
\section{Directions to move}
\begin{enumerate}
	\item Look at generalising $p_i^r$ for general r.
	\item Generalizing to $q$ analog of cyclic group.
	\item Try relating $p_i,q_i.$
	\item Coding which groups $G$ we have $p_i=q_i.$
	\item When are $p_i = q_i.$
	\item Try to compute $q_i.$
	\item Look at simple groups, and maybe solvable groups, try quotienting by normal subgroups?
	\item Are there any ways to combine $G_1,G_2$ where $G_i$ are groups with $p_i = q_i.$
	\item Are there some characterisations of groups with $q_i,p_i.$
	\item How to use sage, what can we do with groups?
	\item Which edge poset definition do we want? Do we include edges containing $y$ or exclude them?
	\item Look at $B_n(q).$ 
	\item Look at generalizing $F_r(B_n)$ to arbitrary posets
	\item Try relating Wilson's Normal Form to our posets?
\end{enumerate}

\section{Edge Functor}
\begin{rem}
We assume all posets are ranked posets, and $G$ actions are rank preserving, order preserving actions.
\end{rem}

\begin{defn}
A poset is $B_k$ full if whenever it contains a vertex $v$ and $p$ vertices above $v,$ then it contains a $p$ dimensional hypercube containing $v$.
\end{defn}

\begin{lem}
$B_n$ is $B_k$ full for all $k.$ Quotients of $B_n$ are $B_k$ full.
\end{lem}
\begin{proof}

\end{proof}



\begin{defn}
Define the poset category $\mathcal P_r,$ where $P \in \mathcal P_r$ is a ranked poset, and the morphisms $Mor(P,Q)$ are rank preserving, order preserving maps, which send all $B_{r+1}$ to other $B_{r+1}$.
\end{defn}

\begin{defn}
Define the Faces functors (there is one for each r) $\mathcal F_r:\mathcal P_r \rightarrow \mathcal P_r,$ which takes a poset to the poset of its $i$ faces. That is, for each $B_k$ subalgebra of $P_r,$ we associate a point. We say a point $p \lessdot q$ if $p$ and $q$ are nonintersecting boolean subalgebras, and the bottommost point of the cube representing $p$ is right below the bottommost point of the cube representing $q$. It takes a map of posets to the induced map on cubes, by definition of the morphisms in $\mathcal P_r.$ For ease of notation, we shall use $\mathcal F$ for $\mathcal F_1.$
\end{defn}

\section{The Picture for $B_n$}

\begin{thm}
\label{boolean_edge_peck}
For $B_n$ the boolean algebra, $\mathcal F(B_n)$ is unitary peck.
\end{thm}
\begin{proof}
$\mathcal F(B_n)$ is actually just a disjoint union of $n$ copies of $B_{n-1},$ where each copy is indexed as corresponding to the set of pairs $B_{n-1} \cong (\mathcal F(B_n))_{(i)} =\{(y,x)|y\gtrdot x,y/x = i\},$ where $i \in [n].$
\end{proof}
\begin{note}
Let $\Delta(G) \subset G\times G$ denote the diagonal subgroup.
Define $X_G(P) = Ind_{\Delta(G)}^{G\times G}(\mathcal F(V(P)))/(G\times G)$
\end{note}

\begin{note}
Let $V(P)$ denote the graded vector space with basis $p \in P.$ The grading is given by the rank of $p.$
\end{note}

\begin{thm}
\label{xg_isomorphism} We then have an isomorphism of graded vector spaces $\mathcal F(V(P)/G)\cong X_G(P).$
\end{thm}
\begin{proof}
The basis for $F(V(P)/G)$ is exactly given by the edges of $V(P)/G.$ By definition, we have an edge $(Gx,Gy)$ in $V(P)/G$ if and only if and $\exists g,h \in G$ with $gx \lessdot hy.$ Next, $Ind_{\Delta(G)}^{G\times G}(\mathcal F(V(P)))$ is precisely the set of all edges of the form $gx,hy$ for $g,h \in G.$ And hence, we have a natural $G \times G$ action on it. Then, by definition, if we quotient by the $G\times G$ action, we obtain the exact same set of edges as in $\mathcal F(V(P)/G).$ Since ranks are always preserved under these maps, we obtain the claimed isomorphism of graded vector spaces.
\end{proof}

\begin{lem}
\label{unitary_peck_induction}
The poset corresponding the the graded vector space $Ind_{\Delta(G\times G)}^{G\times G}(F\mathcal (V(P)))$ is unitary peck. 
\end{lem}
\begin{proof}
First, by ~\ref{boolean_edge_peck}, we know $\mathcal F(V(P))$ is unitary peck. Then, induction simply makes $|G|$ disjoint copies of $F(V(P)).$ Therefore, we can take the corresponding blcok diagonal raising operators for each disjoint copy, and they obviously provide isomorphisms from level $i$ to $n-i.$
\end{proof}

\begin{thm}
\label{quotient_is_peck}
(Stanley) The quotient of a unitary peck poset by an order preserving, rank preserving group action G, is peck.
\end{thm}

\begin{cor}
\label{xg_peck}
The $X_G(P)$ are vector spaces with an underlying peck poset structure.
\end{cor}
\begin{proof}
By ~\ref{unitary_peck_induction}, we have $Ind_{\Delta(G)}^{G\times G}(\mathcal F(V(P)))$ is unitary peck. But then, since $X_G(P) = Ind_{\Delta(G)}^{G\times G}(\mathcal F(V(P)))/(G\times G),$ by ~\ref{quotient_is_peck} we obtain the poset corresponding to $X_G(P)$ is peck. 
\end{proof}

\begin{cor}
\label{edges_peck}
The poset of edges $\mathcal F(V(P)/G)$ is unitary peck.
\end{cor}
\begin{proof}
By ~\ref{xg_peck} we know $X_G(P)$ is peck, but by ~\ref{xg_isomorphism} we have $\mathcal F(V(P)/G)\cong X_G(P),$ and so $\mathcal F(V(P)/G)$ is peck as well.
\end{proof}

\begin{note}
For any poset $P,$ define $p_i$ to be the number of edges from the ith level set $P_i$ to the $i+1$th level set $P_{i+1},$
\end{note}

\begin{cor}
The sequence $p_i$ is unimodal and symmetric.
\end{cor}
\begin{proof}
By definition, the rank of the ith level set of $\mathcal F(V(P)/G)$ is exactly the number of edges from levels i to i+1 of $P.$ That is $\dim(\mathcal F(V(P)/G)_i) = p_i.$ Since by ~\ref{edges_peck}, $\mathcal F(V(P)/G)$ is peck, it is in particular symmetric and unimodal, and so the $p_i$ are symmetric and unimodal.
\end{proof}







\end{document}